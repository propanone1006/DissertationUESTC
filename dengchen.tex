\documentclass[master, vlined, review]{DissertUESTC}



\begin{document}
	
	% 以下两条命令为高亮示例文档中的关键内容而设,正式撰写时切不可使用
	\newcommand{\shad}[1]{\textcolor{DodgerBlue}{\ttfamily #1}}
	\newcommand{\shadcmd}[1]{\shad{$\backslash$#1}}


	% 下方命令允许跨页排版包含多个行间公式的公式组,可选参数可取值为1,2,3,4,越大表示跨页倾向越高
	% \allowdisplaybreaks[3]


	% 当论文中某节的内容接近填满页面且其下紧随几项标题时,LaTeX更倾向于在后续的标题前分页,并且纵向拉伸当前页内容的段间距,以实现纵向分散对齐,也就是很多人在问的现象。这并不是模板bug,而是LaTeX特性。
	% 出现这种情况的本质原因是用户的内容,尤其是在页面中有些图、表、标题的时候,它们的高度很可能不是正文行距的整数倍,那必然就会出现这种问题。
	% 如果你觉得Word从上到下直接堆叠内容,然后在页尾留下明显空白的处理方式更合你意,那就使用下方的\raggedbottom命令
	% \raggedbottom  % 此命令可让LaTeX像Word那样直接堆叠页面内容,而不再默认拉伸段间距,代价是页尾可能会有明显空白
	

	% \setconfidential命令用于设置论文封面中的“密级信息”。慎用!!!学生个人不应随意将论文定性为涉密,需要先经过审批。此命令必须先于\uestccover使用才有效
	% 命令参数:\setconfidential(<信息右上角与纸张左、上边界的距离,以逗号分隔>)[<字体格式>]{<密级>}{<保密期限>}。可选参数默认值:(18cm,2cm)[\zihao{4}\bfseries]
	% \setconfidential{密级}{保密期限}  % 非涉密论文一定要注释掉该命令


	% 封面示例————“双学位学士”以外的学位论文
	%	\uestccover[<学院名称排版风格>]{<论文题目>}{<学科专业>}{<学号>}{<作者姓名>}{<指导教师>}{<教师职称>}{<学院>} % 单学位学士、研究生通用
	% \uestccover{}{}{}{}{}{}{}  % 空封面
	\uestccover{面向大模型推理服务的攻击检测研究}
				{信息与通信工程}
				{081000}
				{邓\hspace{1em}晨}
				{于富财}
				{副教授}%{终南山古墓派}
				{信息与通信工程学院}  % 默认将过长的学院名称压缩至下划线宽度

	
	% 中文扉页,仅研究生用
	% \DegLv{}  % 清空文档类选项设置的<申请学位级别>
	% \uestczhtitlepage  % 空白中文扉页

	\ClsNum{TN828.6}  % \ClsNum{<分类号>}
	\ClsLv{公开}  % \ClsLv{<密级>}
	\UDC{621.39}  % \UDC{<UDC号>}
	\DissertationTitle{关于我的杀父仇人疑似是名震天下的大侠时\\该如何报仇}  % \DissertationTitle{<题名>}
	\Author{杨过}  % \Author{<作者姓名>}
	\Supervisor{小龙女}{掌门}{古墓派}{活死人墓}  % \Supervisor{<指导教师>}{<职称>}{<单位名称>}{<单位地址>}
	% % 副导师信息,无则注释
	% \AssociateSupervisor{洪七公}{前帮主}{丐帮}{襄阳}  % \AssociateSupervisor{<副导师名称>}{<职称}>{<单位名称>}{<单位地址>}

	% \DegLv{西狂}  % \DegLv{<申请学位级别>},该信息由文档类选项自动确定,需修改默认内容时使用,否则注释即可
	\Major{玉女素心剑法}  % \Major{<学科专业>}
	\Profield{剑道}  % \Profield{专业学位领域代码},此为专业学位独有,学术学位用户注释即可
	\Date{1959年1月1日}{1961年1月1日}  % \Date{<论文提交日期>}{<论文答辩日期>}
	\Grant{中华武林}{1961年2月2日}  % \Grant{<学位授予单位>}{<学位授予日期>}
	\Reviewer{黄蓉}{一灯大师、老顽童、黄老邪、郭靖、小龙女}  % \Reviewer{<答辩委员为主席>}{<评阅人>}
	\uestczhtitlepage


	% \Supervisor{小龙女}{古墓派末代掌门}{古墓派}{活死人墓}  % \Supervisor{<指导教师>}{<职称>}{<单位名称>}{<单位地址>}
	% % 副导师信息,无则注释
	% \AssociateSupervisor{洪七公}{天下第一帮前帮主}{丐帮}{襄阳}  % \AssociateSupervisor{<副导师名称>}{<职称}>{<单位名称>}{<单位地址>}
	% \Reviewer{黄蓉}{一灯大师、老顽童、黄老邪、郭靖、小龙女、金轮法王、洪七公、欧阳锋}  % \Reviewer{<答辩委员为主席>}{<评阅人>}
	% \uestczhtitlepage  % 过长评阅人文本将被自动压缩至下划线宽度,不会超出页面边界

	% \uestczhtitlepage[compress]  % 在compress选项下,“中文扉页”将自动压缩过长职称的字宽,使之不超出官方模板为该内容设定的长度
	

	% 英文扉页,仅研究生用
	% \uestcentitlepage{<文题>}{<专业>}{<学号>}{<作者>}{<导师>}{<副导师>}{<学院>},若无副导师,则将“<副导师>”参数留空即可
	% \uestcentitlepage{}{}{}{}{}{}{}  % 空白英文扉页
	\uestcentitlepage{How to Take Revenge When My Father's Murderer is Suspected to Be a Famous Hero}
					{Jade Lady Soul Sword Technique Jade Lady Soul Sword Technique}
					{1182000}
					{Yang Guo}
					{Grandmaster Dragondaughter Little}{Grandmaster Northern Beggar}
					{Ancient Tomb Sect Ancient Tomb Sect Ancient Tomb Sect Ancient Tomb Sect}
	
	% 独创性声明:[<签名宽度>]{<日期>}{<作者签名图片1>}[<作者签名图片2 默认值为作者签名图片1>]{<导师签名图片>},仅研究生用
	\declaration{}{}{}  % 空白独创性声明
	
	% \declaration[3cm]{2024年08月31日}{authsign}{spvrsign}

	% \declaration[3cm]{2024年08月31日}{authsign}[yangguo1]{spvrsign}

	
	
	% 开启中文摘要
	\zhabstract
	
	杨过少年时期母亲染病而亡,随后他便过着四处流浪的生活。

	% 中文关键词
	\zhkeywords{练武;离经叛道;复仇;抗敌;练武;离经叛道;复仇;抗敌;练武;离经叛道;复仇;抗敌;练武;离经叛道;复仇;抗敌;\textcolor{DarkRed}{注意:2025年09月03日修订的\href{https://gr.uestc.edu.cn/xiazai/114/3917}{\color{DarkRed}\CJKunderline*[thickness=0.5bp, format=\color{DarkRed}]{研究生论文撰写规范}}要求改用“;”分隔关键词;(\href{https://www.jwc.uestc.edu.cn/info/1507170256521551874}{\color{DarkRed}\CJKunderline*[thickness=0.5bp, format=\color{DarkRed}]{双学位}})\href{https://www.sice.uestc.edu.cn/info/1140/14689.htm}{\color{DarkRed}\CJKunderline*[thickness=0.5bp, format=\color{DarkRed}]{学士论文撰写规范}}尚未调整,仍使用“,”分隔关键词}}
	
	
	% 开启英文摘要
	\enabstract
	
	When Yang was a teenager, his mother contracted a disease and died, 
	
	% 英文关键词
	\enkeywords{Martial arts;Apostasy;Revenge;Fighting against the enemy;Martial arts;Apostasy;Revenge;Fighting against the enemy}

	
	\tableofcontents  % 主目录,必要
	
	% \listoffigures  % 【图目录】图多则放,反之不放
	
	% \listoftables  % 【表目录】表多则放,反之不放
	
	\listofsymbs  % 生成主要符号表标题,需要额外维护符号表内容,当前符号表页码需要自行维护
	\begin{symbtable}
		a & 加速度(acceleration) & 1 \\
		A & 振幅(amplitude)、面积(Area)、磁场矢量势(magnetic vector potential) & 2 \\
		B & 磁场、磁感应强度、核结合能 & 3 \\
		c & 真空中光速 & 4 \\
		C & 比热容(heat capacity)、电容 & 5 \\
		d & 长度(distance)、直径(diameter)、微分(differential,如dx)& 6 \\
		D & 电位移矢量(electric displacement) & 7 \\
	\end{symbtable}
	
	
	
	% 打印缩略词表,\printnomenclature[<英文缩写宽度>](<中文全称宽度>)
	% 这里可以乱序,输出后会自动按照缩略词的英文全称进行排序
	\printnomenclature
	\nomchn{LP}{Linear Programming}{线性规划}  % 创建缩略词条目,% \nomchn{<缩略词>}{<英文全称>}{<中文全称>}
	\nomchn{PLE}{Path Loss Exponent}{路径损失指数}
	\nomchn{QoS}{Quality of Service}{服务质量}
	\nomchn{SLA}{Service Level Agreement}{服务水平协议}
	\nomchn{NLP}{non-linear programming}{非线性规划}
	\nomchn{4G}{Fourth Generation Mobile Communication Technology}{第四代移动通信技术}
	\nomchn{5G}{Fifth Generation Mobile Communication Technology}{第五代移动通信技术}
	\nomchn{B5G}{Beyond 5G}{超五代移动通信技术}
	\nomchn{NSA}{Non-Standalone}{非独立组网}
	\nomchn{mMIMO}{massive Multiple Input Multiple Output}{大规模多输入多输出}

	% 正文
	\chapter{绪论}

	\chapter{相关技术基础和方法}

	\section{大语言模型LLM推理机制}

	\begin{equation}
		\mathrm{Attention}(Q,K,V)=\mathrm{softmax}\left(\frac{QK^T}{\sqrt{d_k}}\right)V
	\end{equation}

	\begin{figure}[!htb]
		\centering
		\includegraphics[width=0.95\linewidth]{QuestLifecycle}
		\caption{请求的生命周期} \label{fig: 请求的生命周期}
	\end{figure}

	\section{vLLM与内存管理机制}

	\section{异常检测算法基础}


	\chapter{基于多维资源特征的拒绝服务攻击}

	\chapter{基于缓存侧信道的攻击检测研究}
	

	\chapter{总结和展望}
	\section{工作总结}
	\section{后续工作展望}

	
	\acknowledgement
	
	% \null\newpage

	杨过一介狂生,半生漂泊,蒙诸位不弃,屡次仗义相助,此恩此情,铭刻五内。今日斗胆执笔,聊表寸心,若有疏漏,还望海涵。

	
	%% 参考文献部分
	% \nocite{*}% 为了便于在示例文档中展示参考文献而设,正式撰写时需要注释掉
	% \bibliographystyle{DissertUESTC}% 自v25.03.12版本后,参考文献风格文件由用户设定的论文类型选项自动确定,无需在此手动设置
	\bibliography{ref_dengcheng}% 参考文献数据库文件名,注意不要添加后缀名.bib
	
	% 附录起始位置
	\appendix
	
	\chapter{九阴真经原本}
	
	\section{总纲(核心心法)}
	
	\subsection{梵文总纲(后由郭靖、黄蓉译解):}

	原版《九阴真经》的总纲以梵文写成,蕴含武学至高哲理,强调“阴阳互济、刚柔并重”,是化解经中武功戾气的关键。

	关键理念:“天之道,损有余而补不足”(出自《道德经》),主张内力修炼需顺应自然,调和阴阳。

	
	\achievement % 仅研究生用
	
	\section*{发表论文:}
	
	\begin{enumerate}
	    \item \textbf{作者1}, 作者2*, 作者3, 作者4. Domain decomposition method based on integral equation for solution of scattering from very thin, conducting cavity. \emph{IEEE Transactions on Antennas and Propagation}, 2014, 62(10): 5344--5348. (\textbf{CCF评级}, \underline{中科院分区}, IF: 98.8)
	    
		\setcounter{enumi}{98}
	    
		\item \textbf{作者1}, 作者2*, 作者3, 作者4. Domain decomposition method based on integral equation for solution of scattering from very thin, conducting cavity. \emph{IEEE Transactions on Antennas and Propagation}, 2014, 62(10): 5344--5348. (\textbf{CCF评级}, \underline{中科院分区}, IF: 98.8)
	\end{enumerate}
	
	\newpage% 测试多页页眉
	\section*{发明专利:}
	
	\begin{enumerate}
		
		\item \textbf{作者1}, 作者2*, 作者3, 作者4。一种基于xxxxx的真气运转方法: ZL201120846830.0. 2023--02--20.
		
	\end{enumerate}
	
	\section*{参与项目:}
	
	\begin{itemize}
		\item 项目号. 项目名称. 项目级别, 2020.01--2022.12.
	\end{itemize}

\end{document}